\section{Evaluation Methodology}
\label{sec:method}

In order to evaluate this queue against the performance of state-of-the-art designs, I required an implementation of the queue that can be programmed to an FPGA, a testbench to evaluate the performance of the queue and an FPGA device to target and use for the evaluation. For the implementation of the queues, I used a Verilog implementation provided by Professor Somenzi for a the Tagged Up/Down sorter and a Verilog implementation of a min-heap that I developed myself. 

For the FPGA used for evaluation, I chose the Zedboard evaluation board, as this board was accessible to me from prior research. This board includes a Zynq7000 SoC, which incorporates an ARMv7 dual-core CPU and a 7000-series FPGA, allowing for a Linux OS to be able to access and program the FPGA. Using this capability, I instantiated a testbench hardware module in the FPGA along with a particular queue implementation that runs the same test on each queue, referred to as the "runner." This module then returns the results of this test to a program running in the operating system as the number of enqueue operations performed. The software application in the OS accesses this module as a memory-mapped peripheral and records the time to execute the test, as well as the result returned by the module. The runner itself runs a test to enqueue as many elements as possible with incrementing and random priorities, and dequeue these items until the queue is empty, repeatedly for 10,000 iterations. The number of iterations is important, as there is significant overhead when interfacing with the runner module from software, so the FPGA needs to execute a time-consuming operation in order for it to be measurable. Due to time constraints, only a single queue size was used for each evaluation, as the queue size must be specified before synthesis and is time consuming to repeat.

Evaluation of the resource utilization of this queue is much more straight-forward, as utilization metrics are an output of synthesis. Therefore, each design only needs to be synthesized with different queue sizes to determine the utilization metrics. These output metrics are then compared to the published results of the state-of-the-art implementations.