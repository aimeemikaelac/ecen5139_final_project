\section{Discussion}
\label{sec:discussion}

After collecting the data for the resource utilization, it can be seen that LUT resources are much more in demand then flip-flops, but this can not be compared against Huang, et. al.'s implementation. However, the trends show that the Tagged Up/Down sorter does better than the min-heap in almost all cases and better than the state-of-the-art implementations for smaller queue sizes. In addition, the Tagged Up/Down sorter scales better than the min-heap, but both of the state-of-the-art solutions scale even better. It therefore appears that the designers of those queues were more focused on resource utilization then achieving single-cycle performance of the queue.

In terms of performance, the Tagged Up/Down sorter will out-perform the other implementations, since any operation can be done in a single cycle. However, this is achieved using a less scalable design, so a large queue size may not be implementable. The min-heap implementation is similar, but has even worse scaling properties, so the queue size that it can practically support is very limited. These two implementations achieve the same performance, but the min-heap can only be implemented in this FPGA using a very small queue size, so is not practical for use.